\chapter{Background}


\origpage{11}

%%-------------------------------------
\section{Propositional Formulas}\label{sec:1-1}


We define the language common to all the propositional modal logics we will be considering.
% 
(Notation varies from book to book, by the way.)



We begin by specifying our \underline{\textit{alphabet}}.
\begin{enumerate}[itemsep=5pt,parsep=5pt,leftmargin=3em,topsep=5pt,label=\arabic*)] %% or label = \alph*, \roman*
    \item 
    We assume we have the following propositional connectives, modal operators, and logical constants:
    \begin{center}
        \renewcommand{\arraystretch}{1.3} % 行距
        \renewcommand{\arraycolsep}{1.2em} % 列距
    \begin{tabular}{ll}
        $\land$ & (and) \\
        $\lor$ & (or) \\
        $\lnot$ & (not) \\
        $\supset$ & (implies) \\  
        $\Box$ & (necessary) \\
        $\Diamond$ & (possible) \\
        $\bot$ & (falsehood) \\
        $\top$ & (truth-hood) \\
    \end{tabular}
    \end{center}

    \item 
    We suppose we have an infinite supply of \underline{\textit{propositional variables}}.
    % 
    If the reader chooses to understand this as meaning a \underline{\textit{countable list}}, 
    we have no objection, 
    though all our results hold for languages of arbitrary cardinality.



    We use ``$P$'', ``$Q$'', ``$R$'', $\dots$ to stand for propositional variables.
    % 
    Until we get to quantifiers in Chapter 7[][],
    ``variable'' will mean propositional variable.


    \item 
    We have left and right parentheses, as punctuation symbols.
\end{enumerate}
% 
This completes our alphabet.


\origpage{12}
By an atomic formula we mean any variable, $\bot$ or $\top$.



The notion of \underline{\textit{formula}} is given by the following four recursive rules:
\begin{enumerate}[itemsep=5pt,parsep=5pt,leftmargin=3em,topsep=5pt,label=\arabic*)] %% or label = \alph*, \roman*
    \item Any atomic formula is a formula.

    \item If $A$ is a formula, so is $\neg A$.

    \item If $A$ and $B$ are formulas, so are $(A \land B)$, $(A \lor B)$, $(A \supset B)$.

    \item If $A$ is a formula, so are $\Box A, \Diamond A$.
\end{enumerate}


For this definition of formula, 
a \underline{\textit{unique decomposition theorem}} can be shown.
% 
It says that for every formula $X$,
exactly one of the following holds:
\begin{enumerate}[itemsep=5pt,parsep=5pt,leftmargin=3em,topsep=5pt,label=\arabic*)] %% or label = \alph*, \roman*
    \item $X$ is atomic,

    \item there is a \underline{\textit{unique}} formula $Y$ and a \underline{\textit{unique}} symbol $S$ from the list $\neg,\Box,\Diamond$ such that $X = SY$,

    \item there are \underline{\textit{unique}} formulas $Y$ and $Z$ and a \underline{\textit{unique}} connective $B$ chosen from the list $\land,\lor,\supset$ such that $X = (YBZ)$.
\end{enumerate}






We do not prove this here, 
but we make use of it many times, 
without explicit mention.



We define the notion of \underline{\textit{immediate subformula}} as follows:
\begin{enumerate}[itemsep=5pt,parsep=5pt,leftmargin=3em,topsep=5pt,label=(\arabic*)] %% or label = \alph*, \roman*
    \item Atomic formulas have no immediate subformulas.

    \item $X$ is an immediate subformula of $\neg X$, $\Box X$ and $\Diamond X$.

    \item $X$ and $Y$ are immediate sub formulas of $(X \land Y)$, $(X \lor Y)$, and $(X \supset Y)$.
\end{enumerate}


The \textit{proper-subformula-of} relation is the transitive closure of the immediate-subformula-of relation.
% 
That is,
for formulas $X$ and $Y$, 
$X$ is a proper subformula of $Y$ if there is a sequence, 
beginning with $X$, ending with $Y$, 
and in which each term is an immediate subformula of the next.


\origpage{13}
By a \textit{subformula} of $X$ we mean any proper subformula of $X$, or $X$ itself.


By the \textit{degree} of a formula we mean the number of occurrences of the symbols $\land,\lor,\supset,\neg,\Box,\Diamond$ it contains.
% 
We will often prove things about formulas by doing an induction on their degrees. 
% 
Note that the degree of any proper subformula of $X$ must be less than the degree of $X$.
% 
This is the central fact upon which all such induction arguments are based.



Generally, 
when displaying formulas, 
we will abbreviate them by leaving off the outermost parentheses.
% 
Also we will use square and curly brackets as notational variants of parentheses, 
to make reading easier.



\paragraph{Concluding Note:}

As an alternative to the above, 
we could have defined \textit{formulas} as follows.
% 
Let $\mathbb{W}$ be the set of all words (finite strings) over our alphabet. 
% 
% 
If \rotatebox[origin=c]{180}{$\pitchfork$} is a map from the power set of $\mathbb{W}$ to itself, 
\rotatebox[origin=c]{180}{$\pitchfork$} is called an operator on $\mathbb{W}$.\footnote{
    [][] Although I know the symbol $\rotatebox[origin=c]{180}{$\pitchfork$}$ maybe $\psi$ in Fitting's original book, 
    but that symbol is pretty cool!
}
% 
Now, 
let $\Phi$ be the following particular operator on $\mathbb{W}$:

For a set of words $\mathbb{P}$,
$\Phi(\mathbb{P})$ is the set consisting of:
\begin{enumerate}[itemsep=5pt,parsep=5pt,leftmargin=3em,topsep=5pt,label=(\arabic*)] %% or label = \alph*, \roman*
    \item all atomic formulas,

    \item $\neg X,\Box X$ and $\Diamond X$ for all $X \in \mathbb{P}$, 

    \item $(X \land Y)$, $(X \lor Y)$ and $(X \supset Y)$ for all $X,Y \in \mathbb{P}$.
\end{enumerate}
This operator is \textit{monotone}, 
that is,
\[
    \mathbb{P} \subseteq \mathbb{Q} 
    \quad\Longrightarrow\quad
    \Phi(\mathbb{P}) \subseteq \Phi(\mathbb{Q}).
\]


It is a general fact that monotone operators always possess \textit{smallest fixed points}.
% 
Well, 
the set of formulas is the smallest fixed point of the operator $\Phi$  constructed above. 
% 
That is, 
the set of formulas is that set $\mathbb{F}$ such that
\origpage{14}
\[
    \Phi(\mathbb{F}) = \mathbb{F} 
    \qquad
    \text{and} 
    \qquad 
    \Phi(\mathbb{P}) = \mathbb{P} ~~\Longrightarrow~~ \mathbb{F} \subseteq \mathbb{P}.
\]


Of course this amounts to saying the set of formulas is the smallest set $\mathbb{F}$ containing the atomic formulas and closed under the operations 
\begin{enumerate}[itemsep=5pt,parsep=5pt,leftmargin=3em,topsep=5pt,label=\arabic*)] %% or label = \alph*, \roman*
    \item  
    if $X \in \mathbb{F}$ then $\neg X$, $\Box X$ and $\Diamond X$ are in $\mathbb{F}$.

    \item 
    if $X,Y \in \mathbb{F}$ then $(X \land Y)$, $(X \lor Y)$ and $(X \supset Y)$ are in $\mathbb{F}$.
\end{enumerate}



It is also a general fact that one can prove things about the members of the smallest fixed point of a monotone operator by a kind of generalized induction, as follows.
% 
Suppose \rotatebox[origin=c]{180}{$\pitchfork$} is a monotone operator, 
and suppose that whenever all the members of $\mathbb{R}$ have property $P$, 
so do all the members of \rotatebox[origin=c]{180}{$\pitchfork$}($\mathbb{R}$).
% 
Then every member of the smallest fixed point $\mathbb{F}$ of \rotatebox[origin=c]{180}{$\pitchfork$} will have property $P$.
%
%  
For particular operator $\Phi$ defined above, 
this means we can show every formula has a property $P$ provided we can show 
\begin{enumerate}[itemsep=5pt,parsep=5pt,leftmargin=3em,topsep=5pt,label=(\arabic*)] %% or label = \alph*, \roman*
    \item every atomic formula has property $P$,

    \item if $X$ has property $P$, so do $\neg X$, $\Box X$, and $\Diamond X$,

    \item if $X$ and $Y$ have property $P$, so do $(X \land Y)$, $(X \lor Y)$, and $(X \supset Y)$.
\end{enumerate}


As a matter of fact, 
proving things about all formulas using the above technique is essentially the same as doing an induction on degree.
% 
The reader is thus free to interpret our proofs in this book in whichever manner seems most congenial.



%%-------------------------------------
\section{Models}\label{sec:1-2}





%%-------------------------------------
\section{A Unifying Notation}\label{sec:1-3}



%%-------------------------------------
\section{Classical Semantic Tableaus}\label{sec:1-4}
\origpage{29}


In Smullyan [1968] a tree proof procedure for Classical logic was extensively developed. 
% 
It has its origins in Gentzen [1935] and traces its descent through Hintikka [1955], Beth [1959] and Smullyan [1966]. 
% 
It has the double virtue of being easy to prove things with, 
and perspicuous to prove things about. 
% 
In this section we briefly recall the features of the propositional part of it, 
then in the next chapter we adapt it to certain modal logics.




